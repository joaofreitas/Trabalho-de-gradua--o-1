\chapter{Fundamentos básicos}

Este capitulo apresenta os principais conceitos e definições necessários para o entendimento deste trabalho. A seção 2.1 apresenta definições de sistemas multiagentes
A seção 2.2 apresenta uma breve explicação sobre a arquitetura da plataforma JAMA, bem como o seu funcionamento.

\section{Sistemas Multiagentes}

\subsection{Inteligência artificial}

Antes de explicarmos o conceito de sistemas multiagentes (SMA), é necessário mostrar conceitos que são base para o entendimento de SMA. Inicia-se apresentando alguns conceitos de Inteligencia Artificial (IA). De acordo com~\cite{poole98} identificamos que a definição de IA pode variar em duas dimensões principais. Usando a definição de sistemas computacionais que agem racionalmente temos:

\begin{quote}
\emph{Computational Intelligence is the study of the design of intelligent agents.}
\end{quote}

Nessa definição, é importante ressaltar que o agente é uma entidade que age, esperando-se que ele seja provido de características que o diferencie de simples programas.

Com o crescimento dos estudos relacionado a este campo, a inteligência artificial ganhou várias áreas de atuação e resolução de problemas no nosso cotidiano. Um dos problemas é a necessidade de executar aplicações que resolvem problemas de alta complexidade. Essas aplicações podem exigir um hardware muito caro para a execução, ou então pode-se usar a abordagem de distribuí-la em vários computadores que dividem a sua execução. É justamente onde entra a inteligência artificial distribuída: São sistemas que são compostos por vários agentes coletivos, ou seja, distribuem o trabalho uns com os outros. Cada agente pode possuir uma capacidade diferente, sendo possível realizar a tarefa de modo paralelo. 

\subsection{Agente}

De acordo com~\cite{wooldridge95}, agentes são entidades (reais ou virtuais) que funcionam de forma autônoma em um ambiente, ou seja, não necessitam de intervenção humana para realizar processamento. Esse ambiente de funcionamento do agente geralmente contém vários outros agentes e é possível a comunicação entre eles através do ambiente por meio de troca de mensagens. Em geral o funcionamento de agentes acontece de forma a perceberem o ambiente em que estão por meio de sensores, fazem análises com base nessa interação inicial e por fim podem agir sobre o ambiente de forma a modifica-lo por meio de efetuadores.

Agentes racionais seguem o princípio de racionalidade básico: sempre objetivam suas ações pela escolha da melhor ação possível segundo seus conhecimentos. Logo é possível inferir que a ação de um agente nem sempre alcança o máximo desempenho, sendo desempenho o parâmetro definido para medir o grau de sucesso da ação de um agente com base nos seus objetivos.

Exemplo de agente

\section{Dependabilidade}

A dependabilidade de software é a propriedade medida em um sistema computacional A que pode justificadamente confiar em um serviço prestado por um sistema B. Mais especificamente,
em um sistema multiagente, é a propriedade que podemos medir sobre a confiança que podemos mensurar em um agente dado que ele roda em um ambiente descentralizado. Devido a complexidade
do ambiente, existem diversos desafios de acordo com~\cite{hoffman08} no projeto desse ambiente. Em~\cite{algirdas04} são definidos as principais propriedades relacionadas
a dependabilidade, sendo um dos objetivos deste trabalho a identificação destas propriedades na plataforma JAMA.

\section{JAMA}

O desenvolvimento completo da plataforma JAMA está descrita em~\cite{parise11}. Seu objetivo foi desenvolver uma plataforma de sistemas multiagentes descentralizado
e fracamente acoplado, garantindo a comunicação dos agentes com os outros agentes e com os serviços do ambiente. Essa plataforma foi desenvolvida na linguagem Javae outros
frameworks de softwares distrubuidos. O trabalho porém carece de uma análise de dependabilidade, envolvendo os seus componentes e integração dos mesmos. Faz-se necessário
então um estudo mais aprofundado na plataforma, sendo esse portanto o foco principal deste trabalho.




