\chapter{Fundamentos básicos}

Este capitulo apresenta os principais conceitos e definições necessários para o entendimento deste trabalho. A seção 2.1 apresenta definições de sistemas multiagentes
A seção 2.2 apresenta uma breve explicação sobre a arquitetura da plataforma JAMA, bem como o seu funcionamento.

\section{Sistemas Multiagentes}

\subsection{Inteligência artificial}

Antes de explicarmos o conceito de sistemas multiagentes (SMA), é necessário mostrar conceitos que são base para o entendimento de SMA. Inicia-se apresentando alguns conceitos de Inteligencia Artificial (IA). De acordo com~\cite{poole98} identificamos que a definição de IA pode variar em duas dimensões principais. Usando a definição de sistemas computacionais que agem racionalmente temos:

\begin{quote}
\emph{Computational Intelligence is the study of the design of intelligent agents.}
\end{quote}

Nessa definição, é importante ressaltar que o agente é uma entidade que age, esperando-se que ele seja provido de características que o diferencie de simples programas.

\subsection{Agente}

Inteligencia artificial é a área da Ciência da Computaçao onde estuda-se a inteligência de maneira teórica e experimental. Nela, uma das sub-áreas
de IA é o estudo de agentes inteligentes e sistemas multiagentes. Sistemas multiagentes são sistemas compostos por vários agentes que, de acordo com~\cite{wooldridge95}
cada agente é uma entidade virtual a qual é capaz de perceber o ambiente onde está executando, realiza um processamento
sem a intervenção dos outros agentes ou de pessoas e então modifica esse ambiente. Os agentes agem sobre um ambiente através de sensores e agem sobre o mesmo através
de efetuadores. A interação entre esses agentes ocorre apenas por troca de mensagens do componente para o ambiente e este encarrega-se de enviar a mensagem para o ambiente.

\section{Dependabilidade}

A dependabilidade de software é a propriedade medida em um sistema computacional A que pode justificadamente confiar em um serviço prestado por um sistema B. Mais especificamente,
em um sistema multiagente, é a propriedade que podemos medir sobre a confiança que podemos mensurar em um agente dado que ele roda em um ambiente descentralizado. Devido a complexidade
do ambiente, existem diversos desafios de acordo com~\cite{hoffman08} no projeto desse ambiente. Em~\cite{algirdas04} são definidos as principais propriedades relacionadas
a dependabilidade, sendo um dos objetivos deste trabalho a identificação destas propriedades na plataforma JAMA.

\section{JAMA}

O desenvolvimento completo da plataforma JAMA está descrita em~\cite{parise11}. Seu objetivo foi desenvolver uma plataforma de sistemas multiagentes descentralizado
e fracamente acoplado, garantindo a comunicação dos agentes com os outros agentes e com os serviços do ambiente. Essa plataforma foi desenvolvida na linguagem Javae outros
frameworks de softwares distrubuidos. O trabalho porém carece de uma análise de dependabilidade, envolvendo os seus componentes e integração dos mesmos. Faz-se necessário
então um estudo mais aprofundado na plataforma, sendo esse portanto o foco principal deste trabalho.




