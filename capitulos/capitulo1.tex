\chapter{Introdução}

Os sistemas atuais tornam-se cada vez mais complexos, necessitando cada vez mais de poder de processamento. Para a execução dessas aplicações são necessários investimentos cada vez mais altos nos hardwares, existindo sempre um fator limitante (custo ou tecnologia). A partir dessa motivação, aplicações começam então a serem projetadas para rodar de forma descentralizada, com componentes e serviços rodando em vários lugares diferentes em uma mesma rede. Quando essas aplicações estão relacionadas à área de Inteligência Artificial, abordamos o campo da Inteligência Artificial Distribuída (IAD).

Uma dificuldade porém é mensurar o grau de confiabilidade dos componentes executando nesse ambiente distribuído. Vários componentes podem ser dependentes uns dos outros, um erro pode ocasionar uma falha abrupta em toda a execução do software. Não é possível confiar 100\% em softwares acreditando que eles sejam livre de erros e falhas. Esse é o objeto de estudo da Dependabilidade de Software.

O objetivo desse trabalho é mensurar em um ambiente distribuído criado na Universidade de Brasília (JAMA) as principais propriedades de dependabilidade, bem como realizar um estudo levantando os componentes mais críticos dessa plataforma. Em seguida, propor um modelo, correção ou atualização para a plataforma visando melhorias tendo em vista a dependabilidade. 
