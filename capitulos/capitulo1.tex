\chapter{Introdução}

Os sistemas atuais tornam-se cada vez mais complexos, necessitando cada vez mais de poder de processamento. Para a execução dessas aplicações são necessários investimentos cada vez mais altos nos hardwares, existindo sempre um fator limitante (custo ou tecnologia).

A partir dessa motivação, aplicações começam então a serem projetadas para rodar de forma descentralizada, com componentes e serviços rodando em vários lugares diferentes em uma mesma rede. Quando essas aplicações estão relacionadas à área de Inteligência Artificial, abordamos o campo da Inteligência Artificial Distribuída (IAD).

É necessário então a distribuição dos agentes em um ambiente onde possam ser executados em várias máquinas físicas. Surge a necessidade então de uma arquitetura de redes robusta e descentralizadas: A redes Peer-to-peer. Com essa tecnologia será possível separar recursos em uma rede, cada integrante dessa rede com uma responsabilidade distinta.

Por esse motivo, foi desenvolvida uma plataforma de agentes distribuídos (JAMA) que visa prover um ambiente com todas essas características, usando a redes Peer-to-peer. Essa plataforma foi desenvolvida usando softwares livres e a linguagem Java, garantindo portabilidade entre vários sistemas operacionais.

Uma dificuldade porém é mensurar o grau de confiabilidade dos componentes executando nesse ambiente distribuído. Vários componentes podem ser dependentes uns dos outros, um erro pode ocasionar uma falha abrupta em toda a execução do software. Não é possível confiar 100\% em softwares acreditando que eles sejam livre de erros e falhas. Esse é o objeto de estudo da Dependabilidade de Software.

O objetivo desse trabalho é mensurar em um ambiente distribuído criado na Universidade de Brasília (JAMA) as principais propriedades de dependabilidade, bem como realizar um estudo levantando os componentes mais críticos dessa plataforma. Em seguida, propor um modelo, correção ou atualização para a plataforma visando melhorias tendo em vista a dependabilidade.

A estrutura desse trabalho consiste na divisão de capítulos visando facilitar a leitura
\begin{itemize}
	\item Capítulo 1, introdução.
	\item Capítulo 2 contém todos os fundamentos teóricos necessários para o desenvolvimento desse trabalho.
	\item Capítulo 3 deverá conter a proposta de solução conténdo metodologia, levantamento de dados e implementação.
	\item Capítulo 4 deverá conter a expermentação e análise de resultados.
	\item Capítulo 5 terá a conclusão e trabalhos futuros.
\end{itemize}
